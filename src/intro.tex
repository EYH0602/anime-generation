\subsection*{Introduction}
\par
Computer vision and image processing are one of the most popular fields of computer science. 
Scientists interpret images as matrices, as images are displaced as a rectangular formation of pixels on our screen.
Academia and industry have made multiple attempts to generate high-quality images
\cite{
    pmlr-v37-gregor15, 
    Qiao_2019_CVPR, 
    NEURIPS2019_1d72310e, 
    NIPS2016_b1301141,
    Goodfellow2020Generative,
    Rombach2022High,
    He_2022_CVPR,
}.
The task of image generation also varies,
but the most studied directions are 
random noise to image,
the text description to the image,
and incomplete image to completed image.
Among these models, Generative Adversarial Networks (GAN) \cite{Goodfellow2020Generative} and Latent Diffusion Models (LDM) \cite{Rombach2022High} are more commonly used for images that are related to art, for example, photography and drawing.

\par
The wave of generative AI is not limited to computer science related academia and industry. 
Jason Allen's artifact created with Midjourney\cite{Midjourney} won first place at the Colorado State Fair's fine arts competition in September 2022 \cite{Teoh2022Art}.
The editor Teoh \cites{Teoh2022Art} further reported that 
``Although the two judges were unaware that Allen's submission was AI-generated, they added that it wouldn't change their decision because they were looking for `how the art tells a story and how it invokes spirit' instead.''
This clearly indicates that the text-to-image generation is capable for storytelling and spirit invoking based on the inputted text description.

\par
Aoi also stated that generative models are about capturing human feelings toward the objects from the text input \cite{Aoi2022Stable}.
Aoi illustrate that both Midjourney and Diffusion model can understand
{\it kawaii} (cute).
{\it Kawaii} is the signature of broadening Japanese culture to the global,
which was created at the time Japanese domestic focus shifted from production to consumption \cite{Yano2013Pink}.
Based on that idea,
% ! use of I
I argue that text-to-image models also have the power to generate anime girl characters,
which is one of the most reflective genres of the Japanese idea of {\it kawaii}.
In fact, there are examples of a customized version of both GAN and Diffusion for anime character generation
\cite{
    Jin2017Towards,
    Ruan2022Anime,
    WaifuDiffusion,
    nizan2020council,
    chong2021gans,
    chong2021jojogan,
}.