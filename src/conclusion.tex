\subsection*{Conclusion}

Although the booming of AI-drawing after the release of LDMs has also increased
the possible of anime drawing by AI,
there are still concerns about AI anime-drawing.
The first issue is that machine learning models often pay more attention on common features
in the training dataset,
instead trying to understand the reason that led to these features.
For example, shojo style is a way to represent the emotion and psychological movements,
but is not the goal of anime drawing.
Secondly, I concludes that there are still remaining ethical issue on AI anime drawing.
These two issues should be considered when integrating AI models into the production.
Furthermore, style learning based on GAN requires high quality data that has similar styles and large in amount,
whereas text-to-image generating models cannot keep the consistency of style of a character,
thus is hard to use in real anime production.
Therefore, I believe that style-controlled LDMs might have great potential in the anime industry if 
both issues, especially ethical issue, can be resolved.
